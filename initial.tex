\documentclass[12pt]{article}
\usepackage[T2A]{fontenc}
\usepackage[utf8]{inputenc}
\usepackage{amssymb,amsmath}
\usepackage{mathrsfs}
\usepackage[english,russian]{babel}
\usepackage{indentfirst}

\renewcommand{\arraystretch}{1.2}

\title{Совместное решение уравнения энергии и уравнений переноса массы для
задачи фильтрации}
\author{Цыбулин Иван}

\newcommand{\pd}[2]{\frac{\partial #1}{\partial #2}}
\let\dividesymbol\div
\renewcommand{\div}{\operatorname{div}}
\newcommand{\grad}{\operatorname{grad}}

\begin{document}
\maketitle

\section{Получение уравнения для давления}

Предлагается следующий метод, в котором исключаются внешние итерации по
температуре. Главными неизвестными являются $c_i, p, E$.

Система уравнений состоит из нескольких уравнений для концентраций $N_i$ и
одного уравнения для полной энергии $E$:
\begin{align}
\label{eq:mass}
\pd{N_i}{t} + \div \mathbf{Q}_i = 0\\
\label{eq:energy}
\pd{E}{t} + \div \mathbf{J} = 0,
\end{align}
где
\begin{align*}
N_i &= \sum_\alpha n_\alpha c_{i,\alpha} \theta_\alpha\\
\mathbf{Q}_i &= \sum_\alpha n_\alpha c_{i,\alpha} 
\mathbf{W}_\alpha\\
E &= \sum_{\alpha} n_\alpha h_{\alpha} \theta_\alpha - p\\
\mathbf{J} &= \sum_{\alpha} n_\alpha h_{\alpha}
\mathbf{W}_\alpha.
\end{align*}
Рассмотрим дополнительно полную энтальпию смеси
\[
H = 
\sum_{\alpha} n_\alpha h_{\alpha} \theta_\alpha = 
\sum_{\alpha} n_\alpha \left(e_{\alpha} + \frac{p}{n_\alpha}\right) \theta_\alpha = 
E + p \sum_{\alpha} \theta_\alpha = E + p.
\]
Также введем молярную энтальпию $\eta = HV = \frac{H}{N}$.
Из уравнений (\ref{eq:mass},\ref{eq:energy}) нужно получить одно уравнение на
давление $p$. Учтем, что $N_i = N_i(c_i, p, \eta)$, поскольку $N_i$ (и $T$) находятся из
решения задачи о фазовом равновесии по значениям $c_i, p, \eta$.
\[
\sum_j \pd{N_i}{c_j} \pd{c_j}{t} + \pd{N_i}{\eta}\pd{\eta}{t} + \pd{N_i}{p} \pd{p}{t} + \div
\mathbf{Q}_i = 0.
\]
Аналогично поступим с уравнением энергии, заменив $E = H - p$:
\[
\sum_j \pd{H}{c_j} \pd{c_j}{t} + \pd{H}{\eta}\pd{\eta}{t} + \pd{H}{p} \pd{p}{t}
- \pd{p}{t} + \div \mathbf{J} = 0.
\]

Заметим, что $N_i = N c_i, H = N \eta$.
\[
V\pd{(N_i, H)}{(c_j, \eta)} = V\begin{pmatrix}
N \delta_{ij} + c_i \pd{N}{c_j} & c_i \pd{N}{\eta}\\
\eta \pd{N}{c_j} & N + \eta \pd{N}{\eta}
\end{pmatrix} 
= \mathbb{I} + \begin{pmatrix}c_i\\\eta\end{pmatrix} \pd{\ln N}{(c_j, \eta)} 
= \mathbb{I} - \begin{pmatrix}c_i\\\eta\end{pmatrix} \pd{\ln V}{(c_j, \eta)} 
\]

Можно показать, что для случая, когда $V$ линейно-однородна по $c_i, \eta$, $(c_i,\,
\eta)$ --- правый собственный вектор с собственным
значением $0$, а любой ему ортогональный - левый со значением $1$. Значит должен
существовать единственный левый собственный вектор с собственным значением $0$,
который будем обозначать $(\omega_i,\, \chi)$.

Линейная комбинация уравнений с коэффициентами $\omega_i, \chi$ дает:
\[
\zeta\pd{p}{t} - \chi\pd{p}{t}
+ \sum_i \omega_i \div \mathbf{Q}_i + \chi \div \mathbf{J} = 0,
\]
где 
\[-\zeta = \left(\sum_i \omega_i N_i + \chi H\right) \pd{\ln V}{p}\]

\section{Дискретизация уравнений массы и энергии}
Запишем полностью неявную дискретизацию уравнений переноса массы и энергии
\begin{align}
&\frac{\hat N_i - N_i}{\tau} + \div {\hat{\mathbf{Q}}_i} = \hat S_i\\
&\frac{\hat E - E}{\tau} + \div {\hat{\mathbf{J}}} = \hat T
\end{align}
Символом $\hat{}$ обозначено значение на верхнем слое по времени. Для решения этих
уравнений применяется итерационный метод Ньютона ($m$ --- номер итерации):
\begin{align*}
N_i^{(m)} &+ \sum_j \left(\pd{N_i}{c_j}\right)^{(m)}\delta c_j + 
\left(\pd{N_i}{\eta}\right)^{(m)} \delta \eta 
+\left(\pd{N_i}{p}\right)^{(m)} (p^{(m+1)} - p^{(m)})\\
&-\tau\div \sum_\alpha 
b_\alpha^{(m)}
c_{i,\alpha}^{(m)}
\varphi_\alpha^{(m)}
\left(\nabla p^{(m+1)} - \rho^{(m)}_\alpha \mathbf{g}\right)\\
&= N_i^{(0)} + \tau S_i^{(m)}
+ \tau \left(\pd{S_i}{p}\right)^{(m)}(p^{(m+1)} - p^{(m)})\\
%
E^{(m)} &+ \sum_j \left(\pd{H}{c_j}\right)^{(m)}\delta c_j + 
\left(\pd{H}{\eta}\right)^{(m)} \delta \eta 
+\left[\left(\pd{H}{p}\right)^{(m)} - 1\right]
(p^{(m+1)} - p^{(m)})\\
&-\tau\div \sum_\alpha 
b_\alpha^{(m)} 
h_{\alpha}^{(m)} 
\varphi_\alpha^{(m)}
\left(\nabla p^{(m+1)} - \rho^{(m)}_\alpha \mathbf{g}\right)\\
&= E^{(0)} + \tau T^{(m)}
+ \tau \left(\pd{T}{p}\right)^{(m)}(p^{(m+1)} - p^{(m)})
\end{align*}

Здесь $b_\alpha$ обозначено произведение $KB n_\alpha$
Возьмем линейную комбинацию этих уравнений с 
множителями $\omega_i, \chi$ определенными ранее. При этом вариации по $c_j,
\eta$ исключаются из результирующего уравнения:
\begin{align*}
&\left[
\sum_i\omega_i^{(m)}\left(\pd{N_i}{p}\right)^{(m)} +
\chi^{(m)}\left(\pd{H}{p}\right)^{(m)} - \chi^{(m)}\right]
(p^{(m+1)} - p^{(m)})\\
&-\sum_i\tau\omega_i^{(m)}\div \sum_\alpha 
b_\alpha^{(m)} 
c_{i,\alpha}^{(m)} 
\varphi_\alpha^{(m)}
\left(\nabla p^{(m+1)} - \rho^{(m)}_\alpha \mathbf{g}\right)\\
&-\tau\chi^{(m)}\div \sum_\alpha 
b_\alpha^{(m)} 
h_{\alpha}^{(m)} 
\varphi_\alpha^{(m)}
\left(\nabla p^{(m+1)} - \rho^{(m)}_\alpha \mathbf{g}\right) = \\
= &\sum_i \omega_i^{(m)}\left(N_i^{(0)} - N_i^{(m)}\right) + 
\chi^{(m)} \left(E^{(0)} - E^{(m)}\right)\\
&+ \sum_i \tau \omega_i^{(m)} S_i^{(m)} + \tau\chi^{(m)} T^{(m)} \\
&+ \tau 
\left[\sum_i \omega_i^{(m)} \left(\pd{S_i}{p}\right)^{(m)}
+ \chi^{(m)}\left(\pd{T}{p}\right)^{(m)} \right]
(p^{(m+1)} - p^{(m)})
\end{align*}

Величина, стоящая в первых квадратных скобках есть $\zeta^{(m)}$.
Обозначим 
\[
\sigma^{(m)} \equiv -\left[\sum_i \omega_i^{(m)} \left(\pd{S_i}{p}\right)^{(m)}
+ \chi^{(m)}\left(\pd{T}{p}\right)^{(m)}\right]
\]
Собирая слева члены только с $p^{(m+1)}$, получаем:

\begin{align*}
&\left(\zeta^{(m)} + \tau\sigma^{(m)}\right) p^{(m+1)}\\
&-\sum_i\tau\omega_i^{(m)}\div \sum_\alpha 
b_\alpha^{(m)} 
c_{i,\alpha}^{(m)} 
\varphi_\alpha^{(m)}
\grad p^{(m+1)} \\
&-\tau\chi^{(m)}\div \sum_\alpha 
b_\alpha^{(m)} 
h_{\alpha}^{(m)} 
\varphi_\alpha^{(m)}
\grad p^{(m+1)} \\
{}={} &\left(\zeta^{(m)} + \tau\sigma^{(m)}\right) p^{(m)}\\
&-\sum_i\tau\omega_i^{(m)}\div \sum_\alpha 
b_\alpha^{(m)} 
c_{i,\alpha}^{(m)} 
\varphi_\alpha^{(m)}
\rho^{(m)}_\alpha \mathbf{g}\\
&-\tau\chi^{(m)}\div \sum_\alpha 
b_\alpha^{(m)} 
h_{\alpha}^{(m)} 
\varphi_\alpha^{(m)}
\rho^{(m)}_\alpha \mathbf{g} \\
&+\sum_i \omega_i^{(m)}\left(N_i^{(0)} - N_i^{(m)} + \tau S_i^{(m)}\right) + 
\chi^{(m)} \left(E^{(0)} - E^{(m)} + \tau T^{(m)}\right)
\end{align*}

Таким образом, вклад источников проявляется в добавках
\begin{align}
\tilde N_i^{(m)} &= N_i^{(0)} + \tau S_i^{(m)}\\
\tilde E^{(m)} &= E^{(0)} + \tau T^{(m)}\\
\tilde \zeta^{(m)} &= \zeta^{(m)} + \tau\sigma^{(m)}
\end{align}
а уравнения остаются формально такими же, как и для случая без источников.

Уравнение для давления:
\begin{align*}
&\tilde \zeta^{(m)} p^{(m+1)}\\
&-\sum_i\tau\omega_i^{(m)}\div \sum_\alpha 
b_\alpha^{(m)} 
c_{i,\alpha}^{(m)} 
\varphi_\alpha^{(m)}
\grad p^{(m+1)} \\
&-\tau\chi^{(m)}\div \sum_\alpha 
b_\alpha^{(m)} 
h_{\alpha}^{(m)} 
\varphi_\alpha^{(m)}
\grad p^{(m+1)} \\
{}={}&\tilde \zeta^{(m)} p^{(m)}\\
&-\sum_i\tau\omega_i^{(m)}\div \sum_\alpha 
b_\alpha^{(m)} 
c_{i,\alpha}^{(m)} 
\varphi_\alpha^{(m)}
\rho^{(m)}_\alpha \mathbf{g}\\
&-\tau\chi^{(m)}\div \sum_\alpha 
b_\alpha^{(m)} 
h_{\alpha}^{(m)} 
\varphi_\alpha^{(m)}
\rho^{(m)}_\alpha \mathbf{g} \\
&+\sum_i \omega_i^{(m)}\left(\tilde N_i^{(m)} - N_i^{(m)}\right) + 
\chi^{(m)} \left(\tilde E^{(m)} - E^{(m)}\right)
\end{align*}

Уравнение для концентраций:
\[
\frac{N_i^{(m+1)} - \tilde N_i^{(m)}}{\tau} + \div \mathbf{Q}_i^{(m+1)} = 0
\]

Уравнение для энергии:
\[
\frac{E^{(m+1)} - \tilde E^{(m)}}{\tau} + \div \mathbf{J}^{(m+1)} = 0
\]

\section{Задача о фазовом равновесии}

\begin{table}[ht!]
\centering
\begin{tabular}{c|c|c|c|c|c|c}
\multicolumn{1}{c}{$\diagdown$}&
\multicolumn{1}{c}{L (нефть)}&
\multicolumn{1}{c}{G (газ)}&
\multicolumn{1}{c}{W (вода)}&
\multicolumn{1}{c}{S (скелет)}&
\multicolumn{1}{c}{K (кероген)}&
\multicolumn{1}{c}{}\\
\cline{2-6}
$\mathrm{H_2O}$& & $c_0 - \omega$ & $\omega$ & & & $c_0$\\
\cline{1-6}
$\mathrm{C_5H_{12}}$& $\lambda_1$ & $c_1 - \lambda_1$ & & & & $c_1$\\
\cline{2-6}
$\mathrm{C_{10}H_{22}}$& $\lambda_2$ & $c_2 - \lambda_2$ & & & & $c_2$\\
\cline{1-6}
$\mathrm{C_{15}H_{32}}$& $c_3$ & & & & & $c_3$\\
\cline{1-6}
$\mathrm{N_2}$& & $c_4$ & & & & $c_4$\\
\cline{2-6}
$\mathrm{CO_2}$& & $c_5$ & & & & $c_5$\\
\cline{2-6}
$\mathrm{O_2}$& & $c_6$ & & & & $c_6$\\
\cline{1-6}
$\mathrm{Ske}$& & & & $c_7$ & & $c_7$\\
\cline{2-6}
$\mathrm{Ker}$& & & & & $c_8$ & $c_8$\\
\cline{2-6}
\multicolumn{1}{c}{}&
\multicolumn{1}{c}{$\lambda$}&
\multicolumn{1}{c}{$\gamma$}&
\multicolumn{1}{c}{$\omega$}&
\multicolumn{1}{c}{}\\
\end{tabular}
\caption{Распределение компонент по фазам}
\end{table}

Как указывалось выше, требуется решение усложненной задачи о фазовом равновесии,
а именно: при известных $c_i, p, \eta$ найти $T$ и распределение компонент по
фазам.

Предлагается следующее распределение компонент по фазам:
\begin{itemize}
\item Одна компонента $\mathrm{H_2O}$ --- может существовать в фазе $G$ и $W$.
\item $N_{LO}$ компонент --- легкие нефти --- могут существовать в фазах $L$ и
$G$.
\item $N_{HO}$ компонент --- тяжелые нефти --- могут существовать только в фазе
$L$.
\item $N_{G}$ компонент --- нерастворимые газы --- могут существовать только в
фазе $G$.
\item $N_{S}$ компонент --- неподвижные компоненты-фазы --- существуют каждая в
своей фазе.
\end{itemize}

Неизвестными являются $\omega, \lambda_i, T$ --- $N_{LO} + 2$ неизвестных.

Задачу можно сформулировать в виде задачи нелинейного программирования
\begin{align*}
\phi &= \omega \varkappa_0(p,T) + \sum_{i=1}^{N_{LO}} \lambda_i \varkappa_i(p,
T) +\\
&+ \sum_{i=1}^{N_{LO}} L(\lambda_i) - L(\lambda) + \\
&+ L(c_0 - \omega) + \sum_{i=1}^{N_{LO}} L(c_i - \lambda_i) 
- L(s - \lambda - \omega) \rightarrow \min_{\lambda_i, \omega}\\
&\text{При ограничениях}\\
\eta &= 
\sum_{i=0}^{N_{LO}+N_{HO}+N_{G}+N_{S}} c_i h_i(p,T)
-\omega h_{0,G-W}(p,T) 
-\sum_{i=1}^{N_{LO}} \lambda_i h_{i,G-L}(p,T) \\
0& \leq \omega \leq c_0\\
0& \leq \lambda_i \leq c_i, \qquad i = 1,2,\dots,N_{LO}
\end{align*}
Здесь
\begin{itemize}
\item $\phi$ --- с точностью до множителя $RT$ и аддитивной константы потенциал
Гиббса
\item $\varkappa_i(p, T) = \ln K_i(p, T)$ --- логарифмы от констант
равновесия
\item $\lambda_i, \omega$ --- молярные доли жидких
компонент $i$ и жидкой воды
\item $c_i$ --- молярная доля каждого компонента в
жидко-скелетной смеси
\item $L(x) \equiv x \ln x$
\item \begin{align*}
s &\equiv \sum_{i=0}^{N_{LO}+N_{HO}+N_{G}} c_i\\
\lambda &\equiv \sum_{i=1}^{N_{LO}} \lambda_i + \sum_{i = N_{LO} +
1}^{N_{LO}+N_{HO}}c_i\\
\gamma &\equiv s - \lambda - \omega
\end{align*}
\item $h_{i,G-\alpha}(p,T)$ --- удельная энтальпия парообразования
(положительная величина)
\item $h_i(p,T)$ --- молярные энтальпии компонент в ``основной'' фазе. Для воды
и легких нефтей ``основной'' фазой является $G$.
\end{itemize}

Заметим, что задача сформулирована сразу для всех компонент, а не отдельно для
флюидов. Кроме распределения по фазам $c_{i,\alpha}$
и объемных долей $\theta_\alpha = \frac{v_\alpha}{V}$ необходимо найти еще
производные $\pd{V}{c_i}, \pd{V}{\eta}, \pd{V}{p}$. В силу однородной постановки
задачи, $V$ будет однородной функцией относительно $c_i$ и $\eta$ первой степени, то есть
\[\sum_j \pd{V}{c_j}c_j + \pd{V}{\eta}\eta = V.\]

\subsection{Свойства решений}

Задача оптимизации поддается решению в исходной постановке, однако задача о
производных решения оказывается некорректной. Наилучший результат в этой области
гласит, что дифференциал $dV$ есть кусочно-линейный функционал $dc_j$ (фактически,
$V$ не имеет сильной производной, лишь производные по каждому направлению. При
этом имеется конечное число конусов вариаций $dc_j$, в которых производная является сильной).
Практически же, производная перестает существовать в областях, где вариация
параметров может включать и выключать ограничения типа неравенства. В нашей
задаче это соответствует областям, в которых исчезают фазы.

Второй проблемой в дифференцировании по параметрам является изменение структуры
множества ограничений при вариации параметров. В нашей задаче эта проблема
возникает при обращении в ноль концентраций $c_j$. При этом двустороннее
ограничение типа неравенства превращается в ограничение типа равенства, и
производная перестает существовать.

Первой проблемы можно избежать, если запретить задаче иметь решение на границе
области. Практически, это реализуется методом внутренних барьеров, которые
превращают задачу с ограничениями типа равенств и неравенств в задачу только с
равенствами. Более того, для метода внутренней точки наличие логарифмических
барьеров является частью алгоритма. Можно ограничиться некоторым предельным
значением амплитуды барьера, при котором его можно считать мало влияющим на
решение, и этим самым избежать проблемы достижения границы области.

Однако, барьеры невозможно ввести, если ограничение вырождается в равенство.
Поэтому предлагается ограничить минимальное значение $c_i$ некоторой
величиной или в начале производить коррекцию $c_i := \frac{c_i + \epsilon}{1 +
n\epsilon}$. Можно даже корректировать решение обратно линейными поправками, когда будут
посчитаны производные $\pd{\bullet}{c_i}$, однако в этом случае нужно аккуратно
следить за выполнением всех ограничений в задаче.

\subsection{Метод логарифмических барьеров}

Для начала необходимо произвести коррекцию функционала 
\[
\psi = \phi - \varepsilon \left[
\ln \omega + \ln (c_0 - \omega) + 
\sum_{i=1}^{N_{LO}} \Big( \ln \lambda_i
+ \ln (c_i - \lambda_i) \Big) 
\right]
\]
После такой коррекции в области будет существовать локальный минимум,
находящийся вблизи истинного минимума.

Тогда решение задачи необходимо является стационарной точкой функции $\psi$ и
дополнительно удовлетворяет $\eta - h = 0$. Мы не составляем функцию Лагранжа,
поскольку $\psi$ не минимизируется по температуре $T$:
\begin{equation}
0 = \mathbf{F}(\mathbf{x}) \equiv
\begin{cases}
 \pd{\psi}{\omega}\\
 \pd{\psi}{\lambda_i}\\
 \eta - h
\end{cases}
\label{eq:system}
\end{equation}
где $\mathbf{x} = (\omega, \lambda_i, T)$.

Система \eqref{eq:system} далее решается методом Ньютона:
\begin{align}
\Delta \mathbf{x}_k &= -\left(\pd{\mathbf{F(\mathbf{x}_k)}}{\mathbf{x}}\right)^{-1}
\mathbf{F}(\mathbf{x}_k)\\
\mathbf{x}_{k+1} &= \mathbf{x}_k + \alpha_k \Delta \mathbf{x}_k\\
\alpha_k &= \min(1, \beta \alpha_{k, \max})
\end{align}

Чтобы не выйти за границу области определения функции используется ограничитель
шага $\alpha_k \leq 1$. Когда $\alpha_k = 1$ шаг происходит согласно обычному
методу Ньютона. $\alpha_{k, \max} > 0$ --- это максимальное такое число, что
$\mathbf{x}_k + \alpha_{k,\max} \Delta \mathbf{x}_{k}$ еще принадлежит области
$\omega \in [0,c_0], \lambda_i \in [0, c_i], T \geq T_{\min}$. Множитель $\beta
< 1$ (типично $\beta = 0.9 \dividesymbol 0.99$) используется, чтобы новое приближение
было строго внутри области.

Полученное решение легко продифференцировать по параметрам:

\begin{align*}
&\mathbf{F}(\mathbf{x}(\boldsymbol \alpha), \boldsymbol \alpha) = 0\\
&\pd{\mathbf{F}(\mathbf{x}, \boldsymbol \alpha)}{\mathbf{x}} \pd{\mathbf{x}}{\boldsymbol \alpha} +
\pd{\mathbf{F}(\mathbf{x}, \boldsymbol \alpha)}{\boldsymbol \alpha} = 0\\
&\pd{\mathbf{x}}{\boldsymbol \alpha} =
-\left(\pd{\mathbf{F}(\mathbf{x}, \boldsymbol \alpha)}{\mathbf{x}}\right)^{-1}
\pd{\mathbf{F}(\mathbf{x}, \boldsymbol \alpha)}{\boldsymbol \alpha} \\
\end{align*}

\subsubsection{Функция $\mathbf{F}$}

Функция $\mathbf{F}$ состоит из производных по $\omega, \lambda_i$ функции
$\psi$:
\begin{align*}
\phi &= \omega \varkappa_0(p,T) + \sum_{i=1}^{N_{LO}} \lambda_i \varkappa_i(p,
T) +\\
&+ \sum_{i=1}^{N_{LO}} L(\lambda_i) - L(\lambda) + \\
&+ L(c_0 - \omega) + \sum_{i=1}^{N_{LO}} L(c_i - \lambda_i) 
- L(\gamma) - \\
&- \varepsilon \left[
\ln \omega + \ln (c_0 - \omega) + 
\sum_{i=1}^{N_{LO}} \Big( \ln \lambda_i
+ \ln (c_i - \lambda_i) \Big) 
\right]
\end{align*}
и функции $\eta - h$.

\begin{align*}
\pd{\mathbf{\psi}}{\omega} &= \varkappa_0(p, T) - L'(c_0 - \omega) + L'(\gamma) 
- \frac{\varepsilon}{\omega} + \frac{\varepsilon}{c_0 - \omega}\\
\pd{\mathbf{\psi}}{\lambda_i} &= \varkappa_i(p, T) + L'(\lambda_i) - L'(\lambda)
- L'(c_i - \lambda_i) + L'(\gamma) - \frac{\varepsilon}{\lambda_i}
+ \frac{\varepsilon}{c_i - \lambda_i}\\
\eta - h &= 
\eta - \sum_{i=0}^{N_{LO}+N_{HO}+N_{G}+N_{S}} c_i h_i(p,T)
+\omega h_{0,G-W}(p,T) 
+\sum_{i=1}^{N_{LO}} \lambda_i h_{i,G-L}(p,T)
\end{align*}

\subsubsection{Матрица $\pd{\mathbf{F}}{\mathbf{x}}$}

Матрица $\pd{\mathbf{F}}{\mathbf{x}}$ состоит из:
\[
\pd{\mathbf{F}}{\mathbf{x}} = 
\begin{pmatrix}
\pd{{}^2\psi}{\omega^2} & \pd{{}^2\psi}{\omega \partial \lambda_i} & \pd{{}^2
\psi}{\omega \partial T}\\
\pd{{}^2\psi}{\omega \partial \lambda_j} & \pd{{}^2\psi}{\lambda_i \partial
\lambda_j} & \pd{{}^2
\psi}{\lambda_j \partial T}\\
-\pd{h}{\omega} & -\pd{h}{\lambda_i} & -\pd{h}{T}\\
\end{pmatrix}
\]
\begin{align*}
\pd{{}^2\psi}{\omega^2} &= L''(c_0 - \omega) - L''(\gamma) +
\frac{\varepsilon}{\omega^2} + \frac{\varepsilon}{(c_0 - \omega)^2}\\
\pd{{}^2\psi}{\omega \partial \lambda_i} &= - L''(\gamma)\\
\pd{{}^2\psi}{\lambda_i \partial \lambda_j} &=
\delta_{ij} \Big(L''(\lambda_i) + L''(c_i - \lambda_i) +
\frac{\varepsilon}{\lambda_i^2} + \frac{\varepsilon}{(c_i - \lambda_i)^2}\Big) - L''(\lambda) 
- L''(\gamma)\\
\pd{{}^2\psi}{\omega \partial T} &= \pd{\varkappa_0(p, T)}{T} \\
\pd{{}^2\psi}{\lambda_i \partial T} &= \pd{\varkappa_i(p, T)}{T}\\
-\pd{h}{\omega} &= h_{0, G-W}(p, T)\\
-\pd{h}{\lambda_i} &= h_{i, G-L}(p, T)\\
-\pd{h}{T} &=
\omega \pd{h_{0,G-W}(p,T)}{T}
+\sum_{i=1}^{N_{LO}} \lambda_i \pd{h_{i,G-L}(p,T)}{T}
-\sum_{i=0}^{N_{LO}+N_{HO}+N_{G}+N_{S}} c_i \pd{h_i(p,T)}{T}
\end{align*}

\subsubsection{Матрица $\pd{\mathbf{F}}{\mathbf{\boldsymbol\alpha}}$}

Параметрами фазового равновесия $\boldsymbol\alpha$ являются $p, \eta, c_i$.

Производные по $p$:
\begin{align*}
\pd{{}^2\psi}{\omega \partial p} &= \pd{\varkappa_0(p,T)}{p}\\
\pd{{}^2\psi}{\lambda_i \partial p} &= \pd{\varkappa_0(p,T)}{p}\\
\pd{(\eta - h)}{\omega \partial p} &= 
\omega \pd{h_{0,G-W}(p,T)}{p}
+\sum_{i=1}^{N_{LO}} \lambda_i \pd{h_{i,G-L}(p,T)}{p}
-\sum_{i=0}^{N_{LO}+N_{HO}+N_{G}+N_{S}} c_i \pd{h_i(p,T)}{p}
\end{align*}

Производные по $\eta$:
\begin{align*}
\pd{{}^2\psi}{\omega \partial \eta} &= 0\\
\pd{{}^2\psi}{\lambda_i \partial \eta} &= 0\\
\pd{(\eta - h)}{\omega \partial \eta} &= 1
\end{align*}

Производные по $c_0$:
\begin{align*}
\pd{{}^2\psi}{\omega \partial c_0} &= -L''(c_0 - \omega) + L''(\gamma) 
- \frac{\varepsilon}{(c_0 - \omega)^2}\\
\pd{{}^2\psi}{\lambda_i \partial c_0} &= L''(\gamma)\\
\pd{(\eta - h)}{\omega \partial c_0} &= -h_{0}(p,T)
\end{align*}

Производные по $c_j, \quad j = 1,2,\dots,N_{LO}$:
\begin{align*}
\pd{{}^2\psi}{\omega \partial c_j} &= L''(\gamma)\\
\pd{{}^2\psi}{\lambda_i \partial c_j} &= 
-\delta_{ij}\left(L''(c_i - \lambda_i) +
\frac{\varepsilon}{(c_i - \lambda_i)^2}\right)
+ L''(\gamma)\\
\pd{(\eta - h)}{\omega \partial c_j} &= -h_{j}(p,T)
\end{align*}

Производные по $c_j, \quad j = N_{LO}+1,\dots,N_{LO}+N_{HO}$:
\begin{align*}
\pd{{}^2\psi}{\omega \partial c_j} &= 0\\
\pd{{}^2\psi}{\lambda_i \partial c_j} &= -L''(\lambda)\\
\pd{(\eta - h)}{\omega \partial c_j} &= -h_{j}(p,T)
\end{align*}

Производные по $c_j, \quad j = N_{LO}+N_{HO}+1,\dots,N_{LO}+N_{HO}+N_{G}$:
\begin{align*}
\pd{{}^2\psi}{\omega \partial c_j} &= L''(\gamma)\\
\pd{{}^2\psi}{\lambda_i \partial c_j} &= L''(\gamma)\\
\pd{(\eta - h)}{\omega \partial c_j} &= -h_{j}(p,T)
\end{align*}

Производные по $c_j, \quad j =
N_{LO}+N_{HO}+N_{G}+1,\dots,N_{LO}+N_{HO}+N_{G}+N_{S}$:
\begin{align*}
\pd{{}^2\psi}{\omega \partial c_j} &= 0\\
\pd{{}^2\psi}{\lambda_i \partial c_j} &= 0\\
\pd{(\eta - h)}{\omega \partial c_j} &= -h_{j}(p,T)
\end{align*}

\end{document}