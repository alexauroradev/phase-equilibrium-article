\documentclass[12pt]{article}
\usepackage[T2A]{fontenc}
\usepackage[utf8]{inputenc}
\usepackage{amssymb,amsmath}
\usepackage{amsthm}
\usepackage{mathrsfs}
\usepackage[english,russian]{babel}
\usepackage{indentfirst}

\renewcommand{\arraystretch}{1.2}

\title{Расчет фазового равновесия}

\newcommand{\pd}[2]{\frac{\partial #1}{\partial #2}}
\let\dividesymbol\div
\renewcommand{\div}{\operatorname{div}}
\newcommand{\grad}{\operatorname{grad}}
\renewcommand{\epsilon}{\varepsilon}

\newtheorem{note}{Примечание}[section]

%!!!Внимание!!!
%Будем считать, что "компонент" - слово мужского рода ) (а не компонента - женского). В этом случае, правильное склонение таково:
%Падеж	ед.ч.			мн.ч.
%Им.	компоне́нт		компоне́нты
%Р.		компоне́нта		компоне́нтов
%Д.		компоне́нту		компоне́нтам
%В.		компоне́нт		компоне́нты
%Тв.	компоне́нтом	компоне́нтами
%Пр.	компоне́нте		компоне́нтах

\begin{document}
\maketitle

\section{Введение}
%TODO: Написать введение


\section{Двухфазное состояние многокомпонентной смеси}
Рассмотрим смесь из нескольких компонентов при фиксированных давлении $p$ и температуре $T$, находящаяся в жидкой (индекс $L$) и газовой (индекс $G$) фазах. Введем обозначения:
\begin{itemize}
\item $\mu_{i,L}, \mu_{i,G}$ --- химические потенциалы компонента $i$ в фазах $L$ и $G$, соответственно
\item $N_{i,L}, N_{i,G}$ --- количество молей компонента $i$, находящихся в фазах $L$ и $G$, соответственно. При этом $N_L = \sum_i N_{i,L}$, $N_G = \sum_i N_{i,G}$ - общее количество молей вещества, находящегося в каждой из фаз.
\item $x_i = \dfrac{N_{i, L}}{N_L}$, $y_i = \dfrac{N_{i,G}}{N_G}$ --- мольные доли фаз компонента $i$. При этом $ \displaystyle \sum_i x_i = 1 $, $\displaystyle \sum_i y_i = 1 $.
\end{itemize}

Будем также считать, что выражение для химических потенциалов в фазах задаются соотношениями, верными для идеальных растворов \cite{Prigozhin}:

\begin{equation}
\begin{aligned}
\mu_{i,L} = \mu_{i,L}^0 + RT \ln x_i \\
\mu_{i,G} = \mu_{i,G}^0 + RT \ln y_i
\end{aligned},
\label{eq:hid}
\end{equation}
где $\mu_{i,\alpha}^0 = \mu_{i,\alpha}^0(p, T)$ --- химический потенциал чистого вещества $i$, находящегося в фазе $\alpha$ при заданных давлении и температуре.

Потенциал Гиббса для этой системы:

\begin{equation}
\begin{aligned}
\Phi = \sum_\alpha \Phi_\alpha = \sum_{i, \alpha}{N_{i,\alpha} \mu_{i,\alpha}} = \sum_i \left(N_{i, L} \mu_{i, L} + N_{i, G} \mu_{i, G}\right)
\end{aligned}
\label{eq:gibbs}
\end{equation}

Существует несколько условий фазового равновесия системы, которые являются эквивалентными \textbf{[ссылка]}. Первое --- условие равенства хим. потенциалов во всех фазах (в данном случае $\mu_{i, \alpha} = C_i = const$). Второе --- условие достижения минимума потенциала Гиббса $\Phi$ по мольным долям фаз и фазовым составам, с учетом ограничений $\sum_\alpha N_{i, \alpha} = N_i$. Мы будем пользоваться вторым условием, поскольку оно является более универсальным. При условной минимизации потенциала Гиббса отпадает необходимость рассмотрения случаев, связанных с отсутствием фаз, они соответствуют минимуму на границе области.

Введем функцию Лагранжа:
\begin{equation}
\mathscr{L} = \Phi + \sum_i{\xi_i \left[N_i - \sum_\alpha N_{i, \alpha} \right]}
\label{eq:lagr}
\end{equation} 


Условие экстремума функции Лагранжа \eqref{eq:lagr}, учитывая соотношения \eqref{eq:hid}:
\[
\left\{
\begin{aligned}
& \pd{\mathscr{L}}{N_{i, \alpha}} = \mu_{i, \alpha} + N_{i, \alpha} \cdot RT \frac{1}{N_{i, \alpha}} - \xi_i = 0, \qquad \alpha=L, G \\
& \pd{\mathscr{L}}{\xi_i} = N_i - \sum_{\alpha} N_{i,\alpha} = 0 
\end{aligned}
\right.,\quad\forall i
\]

Или:
\begin{equation}
\left\{
\begin{aligned}
& \mu_{i, G}^0 + RT \ln y_i + RT - \xi_i = 0\\
& \mu_{i, L}^0 + RT \ln x_i + RT - \xi_i = 0\\
& N_i - \sum_{\alpha} N_{i,\alpha} = 0
\end{aligned}
\label{eq:lagr2}
\right.,\quad\forall i
\end{equation}

Вычитая в \eqref{eq:lagr2} первое уравнение из второго, получаем, что в состоянии фазового равновесия выполняется условие:

\begin{equation}
\frac{y_i}{x_i} = \exp \left(\frac{\mu_{i, L}^0 - \mu_{i, G}^0}{RT}\right) = K_i(p,T),
\label{eq:lnK}
\end{equation}
где $K_i(p,T) \equiv \dfrac{y_i}{x_i}$ --- константа равновесия для компонента $i$.


\begin{note}
Обратите внимание на то, что выражение \eqref{eq:lnK} можно было получить напрямую из условия $\mu_1 = \mu_2$.
\end{note}


\section{Многофазное многокомпонентное состояние}

\subsection{Постановка задачи и допущения}

Рассмотрим задачу о нахождении фазового равновесия в условиях нефтяного месторождения, разрабатываемого путем водного замещения (возможно также термогазовое воздействие). При этом рассматриваются следующие фазы: $L$ --- жидкая (нефтяная) фаза, $G$ --- газ, $W$ --- водная фаза, а также несколько инертных фаз (не  участвуют в фазовом равновесии).

Предлагается следующее распределение компонентов по фазам:
\begin{itemize}
\item Один компонент $\mathrm{H_2O}$ --- может существовать в фазе $W$ и $G$.
\item $N_{LO}$ компонентов --- легкие нефти --- могут существовать в фазах $L$ и $G$ (например, $\mathrm{C_5H_{12}}$).
\item $N_{HO}$ компонентов --- тяжелые нефти --- могут существовать только в фазе $L$.
\item $N_{G}$ компонентов --- нерастворимые газы --- могут существовать только в фазе $G$.
\item $N_{S}$ компонентов --- инертные компоненты-фазы --- существуют каждая в своей фазе.
\end{itemize}

\begin{note}
Газы, растворимые в нефти, могут моделироваться как легкие нефти. При этом процессы разгазирования могут быть учтены подбором константы равновесия.
\end{note}

Введем обозначения (все молярные доли рассматриваются по отношению к количеству молей $N$ всей смеси):
\begin{itemize}
\item $\lambda, \omega, \gamma$ --- молярные доли в смеси жидкой, водной и газовой фазы соответственно. При этом $\lambda + \omega + \gamma \ne 1$, т.к. учитываются также инертные фазы.
\item В фазе $W$ может находиться только $\mathrm{H_2O}$, а значит $\omega$ является также молярной долей жидкой воды
\item $c_0$ - молярная доля $\mathrm{H_2O}$ в смеси
\item $c_i, \: i = 1 \dividesymbol N_{LO}$ --- молярные доли легких нефтей в смеси
\item $c_i, \: i = N_{LO} + 1 \dividesymbol N_{LO} + N_{HO}$ --- молярные доли тяжелых нефтей в смеси
\item $c_i, \: i = N_{LO} + N_{HO} + 1 \dividesymbol N_{LO} + N_{HO} + N_G$ --- молярные доли нерастворимых газов
\item $c_i, \: i = N_{LO} + N_{HO} + N_G + 1 \dividesymbol N_{LO} + N_{HO} + N_G + N_S$ --- молярные доли неподвижных компонентов. Также являются молярными долями соответствующих фаз. 
\item $\lambda_i, \: i = 1 \dividesymbol N_{LO}$ --- молярные доли легких нефтей в газовой фазе

\end{itemize}

В таблице \ref{t:compphases} приведен пример распределения компонентов по фазам. Компоненты $\mathrm{C_5H_{12}}$ и $\mathrm{C_{10}H_{22}}$ считаются легкими нефтями, компонент $\mathrm{C_{15}H_{32}}$ --- тяжелой нефтью. В смеси присутствуют три нерастворимых газа ($\mathrm{N_2}$, $\mathrm{CO_2}$ и $\mathrm{O_2}$), а также две инертных компоненты ($\mathrm{Sand}$ и $\mathrm{Gran}$)

\begin{table}[ht!]
\centering
\begin{tabular}{c|c|c|c|c|c|c}
\multicolumn{1}{c}{$\diagdown$}&
\multicolumn{1}{c}{L (нефть)}&
\multicolumn{1}{c}{G (газ)}&
\multicolumn{1}{c}{W (вода)}&
\multicolumn{1}{c}{S (песчаник)}&
Gr (гранит)&
$\sum$\\
\cline{2-6}
$\mathrm{H_2O}$& & $c_0 - \omega$ & $\omega$ & & & $c_0$\\
\cline{1-6}
$\mathrm{C_5H_{12}}$& $\lambda_1$ & $c_1 - \lambda_1$ & & & & $c_1$\\
\cline{2-6}
$\mathrm{C_{10}H_{22}}$& $\lambda_2$ & $c_2 - \lambda_2$ & & & & $c_2$\\
\cline{1-6}
$\mathrm{C_{15}H_{32}}$& $c_3$ & & & & & $c_3$\\
\cline{1-6}
$\mathrm{N_2}$& & $c_4$ & & & & $c_4$\\
\cline{2-6}
$\mathrm{CO_2}$& & $c_5$ & & & & $c_5$\\
\cline{2-6}
$\mathrm{O_2}$& & $c_6$ & & & & $c_6$\\
\cline{1-6}
$\mathrm{Sand}$& & & & $c_7$ & & $c_7$\\
\cline{2-6}
$\mathrm{Gran}$& & & & & $c_8$ & $c_8$\\
\cline{1-6}
\multicolumn{1}{c}{$\sum$}&
\multicolumn{1}{c}{$\lambda$}&
\multicolumn{1}{c}{$\gamma$}&
\multicolumn{1}{c}{$\omega$}&
\multicolumn{1}{c}{$c_7$}&
\multicolumn{1}{c}{$c_8$}\\
\end{tabular}
\caption{Пример распределения компонентов по фазам}
\label{t:compphases}
\end{table}

Требуется найти распределение компонентов по фазам, при известных молярных долях компонентов $c_i, \; i = 0 \dividesymbol N_{LO} + N_{HO} + N_{G} + N_{S}$, давлении $p$ и температуре $T$. Неизвестными величинами являются $\{\omega, \lambda_i\}$ --- $N_{LO} + 1$ неизвестная.

Задачу можно сформулировать в виде задачи условной оптимизации:
\begin{equation}
\begin{aligned}
\Phi = &\sum_{i, \alpha} \mu_{i, \alpha} N_{i, \alpha} \rightarrow \min_{N_{i, \alpha}},\\
&\sum_\alpha \frac{N_{i, \alpha}}{N} = c_i
\end{aligned}
\label{eq:problem}
\end{equation}

\subsection{Анализ задачи}

Распишем подробнее потенциал Гиббса, возникающий в задаче \eqref{eq:problem}:

\begin{equation}
\begin{aligned}
\Phi / N = \mu &= (c_0 - \omega) \mu_{\mathrm{H_2O, G}} + \omega \mu_{\mathrm{H_2O, W}} + \\
& + \sum_{i = 1}^{N_{LO}} \left((c_i - \lambda_i) \mu_{i, G} + \lambda_i \mu_{i, L}\right) + \\
& + \sum_{i = N_{LO} + 1}^{N_{LO} + N_{HO}} c_i \mu_{i, L} + \sum_{i = N_{LO} + N_{HO}+ 1}^{N_{LO} + N_{HO} + N_{S}} c_i \mu_{i, \alpha_i}
\end{aligned}
\label{eq:gibbslong}
\end{equation}

Последние два слагаемых в соотношении \eqref{eq:gibbslong} являются постоянными при заданном компонентном составе, давлении и температуре. А значит, они не влияют на решение минимизационной задачи. Преобразуем члены в правой части \eqref{eq:gibbslong}, относящиеся к $\mathrm{H_2O}$ с учетом соотношений для химического потенциала идеального газа и \eqref{eq:lnK}:

\begin{equation}
\begin{aligned}
& (c_0 - \omega) \mu_{\mathrm{H_2O, G}} + \omega \mu_{\mathrm{H_2O, W}} = c_o \mu_{\mathrm{H_2O, G}} + \omega (\mu_{\mathrm{H_2O, W}} - \mu_{\mathrm{H_2O, G}}) = \\
& = c_o \left(\mu_{\mathrm{H_2O, G}}^{0} + RT \ln \frac{c_0 - \omega}{\gamma}\right) + \omega RT \left(\ln K_0 + \ln \frac{\omega}{\omega} - \ln \frac{c_0 - \omega}{\gamma}\right) = \\
& = c_o \mu_{\mathrm{H_2O, G}}^{0} + RT \left(\omega \ln K_0 + (c_0 - \omega) \ln (c_0 - \omega) - (c_0 - \omega) \ln \gamma\right),
\end{aligned}
\label{eq:gibbswater}
\end{equation}
где $K_0$ --- константа равновесия фазового перехода $W-G$ для $\mathrm{H_2O}$.

Аналогично, можно преобразовать выражения для легких нефтей:

\begin{equation}
\begin{aligned}
& \left((c_i - \lambda_i) \mu_{i, G} + \lambda_i \mu_{i, L}\right) = \\
& = c_i \left(\mu_{i, G}^0 + RT \ln \frac{c_i - \lambda_i}{\gamma} \right) + \lambda_i RT \left(\ln K_i + \ln \frac{\lambda_i}{\lambda} - \ln \frac{c_i - \lambda_i}{\gamma}\right) = \\
& = c_i \mu_{i, G}^0 + RT \left(\lambda_i \ln K_i + (c_i - \lambda_i)\ln \frac{c_i - \lambda_i}{\gamma} + \lambda_i \ln \frac{\lambda_i}{\lambda} \right)
\end{aligned}
\label{eq:gibbsliquid}
\end{equation}

Таким образом задача \eqref{eq:problem} с учетом \eqref{eq:gibbslong}, \eqref{eq:gibbswater} и \eqref{eq:gibbsliquid}, записывается в виде:

\begin{equation}
\begin{aligned}
\tilde{\mu} &= \omega \varkappa_0(p,T) + \sum_{i=1}^{N_{LO}} \lambda_i \varkappa_i(p,T) + \sum_{i=1}^{N_{LO}} L(\lambda_i) - L(\lambda) + \\
&+ L(c_0 - \omega) + \sum_{i=1}^{N_{LO}} L(c_i - \lambda_i) - L(s - \lambda - \omega) \rightarrow \min_{\lambda_i, \omega}
\end{aligned}
\label{eq:pproblem}
\end{equation}

При ограничениях:


\begin{equation}
\left\{
\begin{aligned}
0& \leq \omega \leq c_0\\
0& \leq \lambda_i \leq c_i, \qquad i = 1,2,\dots,N_{LO}
\end{aligned}
\right.
\label{eq:restr}
\end{equation}

Здесь
\begin{itemize}
\item $\tilde{\mu}$ --- с точностью до множителя $RT$ и аддитивной константы химический потенциал смеси
\item $\varkappa_i(p, T) = \ln K_i(p, T)$ --- логарифмы констант равновесия
\item $L(x) \equiv x \ln x$
\item $s \equiv \gamma + \lambda + \omega$
\end{itemize}

\begin{note}
Задача сформулирована сразу для всех (в том числе и инертных) компонентов, а не отдельно для флюидов.
\end{note}

\subsection{Энтальпия}
Фазовые переходы жидкость-газ при фиксированном давлении происходят при постоянной температуре. Это обстоятельство приводит к двум важным выводам:
\begin{itemize}
\item В некоторых случаях недостаточно молярных долей компонентов, давления и температуры для того, чтобы однозначно определить фазовый состав.
\item В графиках зависимости фазовых составов от температуры будут содержаться разрывы первого рода.
\end{itemize}

Таким образом требуется дополнительный параметр или соотношение для смеси для однозначного определения фазового состава, а также обеспечения устойчивости. Часто в качестве такого параметра используется молярная энтальпия смеси $\eta$, фигурирующая в уравнении теплового баланса (используется при моделировании разработки месторождения).

В рассмотренной задаче энтальпия имеет вид:
\begin{equation}
\eta = 
\sum_{i=0}^{N_{LO}+N_{HO}+N_{G}+N_{S}} c_i h_i(p,T) -\omega h_{0,G-W}(p,T) -\sum_{i=1}^{N_{LO}} \lambda_i h_{i,G-L}(p,T) 
,
\label{eq:defenth}
\end{equation}
где $h_{i,G-\alpha}(p,T)$ --- удельная энтальпия парообразования (положительная величина), а $h_i(p,T)$ --- молярные энтальпии компонентов в ``основной'' фазе. Для $\mathrm{H_2O}$ и легких нефтей ``основной'' фазой является $G$.

С другой стороны, энтальпия может являться не вспомогательным, а определяющим параметром. Условие фазового равновесия системы и соотношение \eqref{eq:defenth} совместно можно считать нелинейным уравнением на температуру. Его можно трактовать следующим образом: требуется подобрать такую температуру смеси, чтобы после установления фазового равновесия ее молярная энтальпия равнялась заданной величине.

Обратите внимание, что условие \eqref{eq:defenth} \emph{не является} ограничением в задаче \eqref{eq:pproblem}. Соотношение \eqref{eq:defenth} --- это уравнение на $T$, а блок фазового равновесия --- это неявная функция фазовых составов от температуры.

Расчет фазового равновесия обычно является отдельным блоком при моделировании процессов разработки месторождения. Уравнения баланса компонентов и энергии, использующиеся в моделях, в общем случае являются нелинейными. Поэтому кроме фазовых составов и объемных долей фаз $\theta_\alpha = {v_\alpha}/{V}$ (вычисляются по фазовым составам), обычно необходимо найти еще
производные $\pd{V}{c_i}, \pd{V}{\eta}, \pd{V}{p}$. Эти производные используются в итерационном процессе (например, методе Ньютона). Альтернативой точного вычисления производных является использование их разностных аналогов, однако это существенно увеличивает время расчета.

\subsection{Особенности вычисления производных}

Задача вычисления фазовых составов поддается решению в исходной постановке. Однако, если требуется вычисление производных молярного объема, возникает ряд проблем. Наилучший результат в этой области гласит, что дифференциал $dV$ есть кусочно-линейный функционал $dc_i$ (фактически, $V$ не имеет сильной производной, лишь производные по каждому направлению. При этом имеется конечное число конусов вариаций $dc_i$, в которых производная является сильной). По сути, производная перестает существовать в областях, где исчезают фазы.

Второй проблемой в дифференцировании по параметрам является изменение структуры множества ограничений \eqref{eq:restr} при вариации параметров. В нашей задаче эта проблема возникает при обращении в ноль концентраций $c_i$ (случаи, когда какие-то компоненты отсутствуют). При этом двустороннее ограничение типа неравенство превращается в ограничение типа равенство, и производная перестает существовать.

Первой проблемы можно избежать, если запретить задаче иметь решение на границе области. Практически, это реализуется методом внутренних (например, логарифмических) барьеров, которые превращают задачу с ограничениями типа равенств и неравенств в задачу только с равенствами. Более того, для метода внутренней точки наличие логарифмических барьеров является частью алгоритма. Можно ограничиться некоторым предельным значением амплитуды барьера, при котором его можно считать мало влияющим на решение, и этим самым избежать проблемы достижения границы области.

Барьеры невозможно ввести, если ограничение вырождается в равенство. Поэтому предлагается ограничить минимальное значение $c_i$ некоторой величиной или в начале производить коррекцию $c_i = (c_i + \epsilon)/(1 + n\epsilon)$. Это соответствует тому, что все компоненты всегда присутствуют при расчете фазового равновесия. Можно корректировать решение обратно линейными поправками, когда будут посчитаны производные $\pd{}{c_i}$, но в этом случае нужно аккуратно следить за выполнением всех ограничений в задаче.

\section{Метод решения}




\begin{thebibliography}{9}
    \addcontentsline{toc}{section}{\refname}
    \bibitem{Prigozhin} Пригожин И., Дефей Р. Химическая термодинамика //Новосибирск: СО изд.«Наука. – 1966.
\end{thebibliography}

\end{document}