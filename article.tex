\documentclass[12pt]{article}
\usepackage[T2A]{fontenc}
\usepackage[utf8]{inputenc}
\usepackage{amssymb,amsmath}
\usepackage{amsthm}
\usepackage{mathrsfs}
\usepackage[english,russian]{babel}
\usepackage{indentfirst}

\renewcommand{\arraystretch}{1.2}

\title{Расчет фазового равновесия}

\newcommand{\pd}[2]{\frac{\partial #1}{\partial #2}}
\let\dividesymbol\div
\renewcommand{\div}{\operatorname{div}}
\newcommand{\grad}{\operatorname{grad}}
\renewcommand{\epsilon}{\varepsilon}
\renewcommand{\geq}{\geqslant}
\renewcommand{\leq}{\leqslant}

\newtheorem{note}{Примечание}[section]

%!!!Внимание!!!
%Будем считать, что "компонент" - слово мужского рода ) (а не компонента - женского). В этом случае, правильное склонение таково:
%Падеж	ед.ч.			мн.ч.
%Им.	компонент		компоненты
%Р.		компонента		компонентов
%Д.		компоненту		компонентам
%В.		компонент		компоненты
%Тв.	компонентом		компонентами
%Пр.	компоненте		компонентах

\begin{document}
\maketitle

\section{Введение}

Расчет фазового равновесия обычно является отдельным блоком при моделировании различных гидродинамических процессов, например, течении многокомпонентных смесей. Уравнения баланса компонентов и энергии, использующиеся в моделях, в общем случае являются нелинейными и требуют применения итерационных процедур, таких как, метод Ньютона. Поэтому кроме фазовых составов и объемных долей фаз необходимо вычислить производные по параметрам, которые определяют состояние системы и используются в итерационном процессе. Альтернативой точного вычисления производных является использование их разностных аналогов, однако это существенно увеличивает время расчета.

В данной работе фазовое равновесие описывается коэффициентами распределения (константами фазового равновесия), зависящими только от давления и температуры. Компоненты смеси разделяются на два типа: \emph{активные} --- те, которые участвуют в установлении фазового равновесия, и \emph{инертные}, роль которых ограничивается участием в балансе энергии. Существует точка зрения что все компоненты в той или иной мере содержаться во всех фазах. Мы, однако, считаем, что часть компонентов не содержится в некоторых фазах. В этом смысле рассматривается ограниченная задача о фазовом равновесии. Принимается что активные компоненты могут образовать три фазы: жидкую (нефтяную), газовую и водную. Водная фаза состоит только из одного компонента $\mathrm{H_2O}$. Жидкая фаза содержит компоненты, не присутствующие в других фазах (тяжелые нефти), и компоненты, присутствующие в газовой фазе (легкие нефти). Газовая фаза наряду с легкими нефтями сожержит нерастворимые газы (содержащиеся только в газовой фазе) и пары $\mathrm{H_2O}$. Количество фаз, на которое расслаивается смесь, заранее неизвестно и определяется из решения задачи.

Задача о фазовом равновесии, которая рассматривается в данной работе, состоит в следующем. Заданы молярные концентрации всех компонентов смеси, её молярная энтальпия и давление. Требуется определить на какие фазы расслоится смесь, их молярные доли и составы; а также температуру. Для расчета фазового равновесия используется тот факт, что при фиксированных давлении, температуре и составе смеси в состоянии равновесия потенциал Гиббса достигает своего минимума. Так как выражение для потенциала Гиббса неизвестно, используется модельный потенциал Гиббса, обеспечивающий те же константы фазового равновесия, что и в исходной модели, и, как следствие, те же равновесные состояния смеси.

Задача минимизации ставится в пространстве независимых термодинамических степеней свободы: температура и молярные доли $\mathrm{H_2O}$ и легких нефтей. Указанные величины однозначно описывают равновесное состояние системы. Множество, на котором проиходит минимизация ограничено и описывается набором естественных условий неотрицательности молярных долей компонентов. Таким образом ставится задача условной минимизации.

В исходной постановке такая задача нахождения фазового равновесия может быть решена рассмотрением ограниченного числа случаев. Однако, если температура сама является неизвестной величиной, алгоритм расчета фазового равновесия существенно усложняется.

При вычислении производных возникает ряд дополнительных проблем. Во-первых, производные перестают существовать на границе области минимизации. Второй проблемой является изменение структуры множества ограничений при вариации параметров. Эта проблема возникает в случае, когда какие-то компоненты смеси отсутствуют.

Первой проблемы можно избежать, если запретить задаче иметь решение на границе области. Практически, это реализуется методом внутренних (например, логарифмических) барьеров, которые сводят проблему к задаче безусловной минимизации. Более того, для метода внутренней точки наличие логарифмических барьеров является частью алгоритма. Можно ограничиться некоторым предельным значением амплитуды барьера, при котором его можно считать мало влияющим на решение, и этим самым избежать проблемы достижения границы области. Вторая проблема решается добавление к смеси малого количества отсутствующий компонентов и их удаления после расчета фазового равновесия и необходимых производных.

Предлагаемый метод расчета фазового равновесия состоит в следующем:
\begin{enumerate}
	\item Выбор модельного потенциала Гиббса
	\item Модификация потенциала в соответствии с методом логарифмических барьеров
	\item Запись условий необходимых для нахождения стационарной точки модифицированного потенциала при фиксированной температуре
	\item Добавление уравнения энтальпии, как неявного уравнения на температуру
	\item Совместное решение уравнений пунктов 3 и 4 
\end{enumerate}

Предлагаемый метод позволяет единообразно решать все задачи подобного типа, что гарантирует устойчивую работу алгоритма.


\section{Двухфазное состояние многокомпонентной смеси}
Рассмотрим смесь из нескольких компонентов при фиксированных давлении $p$ и температуре $T$, находящаяся в жидкой (индекс $L$) и газовой (индекс $G$) фазах. Введем обозначения:
\begin{itemize}
\item $\mu_{i,L}, \mu_{i,G}$ --- химические потенциалы компонента $i$ в фазах $L$ и $G$, соответственно
\item $N_{i,L}, N_{i,G}$ --- количество молей компонента $i$, находящихся в фазах $L$ и $G$, соответственно. При этом $N_L = \sum_i N_{i,L}$, $N_G = \sum_i N_{i,G}$ - общее количество молей вещества, находящегося в каждой из фаз.
\item $x_i = \dfrac{N_{i, L}}{N_L}$, $y_i = \dfrac{N_{i,G}}{N_G}$ --- мольные доли фаз компонента $i$. При этом $ \displaystyle \sum_i x_i = 1 $, $\displaystyle \sum_i y_i = 1 $.
\end{itemize}

Будем также считать, что выражение для химических потенциалов в фазах задаются соотношениями, верными для идеальных растворов \cite{Prigozhin}:

\begin{equation}
\begin{aligned}
\mu_{i,L} = \mu_{i,L}^0 + RT \ln x_i \\
\mu_{i,G} = \mu_{i,G}^0 + RT \ln y_i
\end{aligned},
\label{eq:hid}
\end{equation}
где $\mu_{i,\alpha}^0 = \mu_{i,\alpha}^0(p, T)$ --- химический потенциал чистого вещества $i$, находящегося в фазе $\alpha$ при заданных давлении и температуре.

Потенциал Гиббса для этой системы:

\begin{equation}
\begin{aligned}
\Phi = \sum_\alpha \Phi_\alpha = \sum_{i, \alpha}{N_{i,\alpha} \mu_{i,\alpha}} = \sum_i \left(N_{i, L} \mu_{i, L} + N_{i, G} \mu_{i, G}\right)
\end{aligned}
\label{eq:gibbs}
\end{equation}

Существует несколько условий фазового равновесия системы, которые являются эквивалентными \textbf{[ссылка]}. Первое --- условие равенства хим. потенциалов во всех фазах (в данном случае $\mu_{i, \alpha} = C_i = const$). Второе --- условие достижения минимума потенциала Гиббса $\Phi$ по мольным долям фаз и фазовым составам, с учетом ограничений $\sum_\alpha N_{i, \alpha} = N_i$. Мы будем пользоваться вторым условием, поскольку оно является более универсальным. При условной минимизации потенциала Гиббса отпадает необходимость рассмотрения случаев, связанных с отсутствием фаз, они соответствуют минимуму на границе области.

Введем функцию Лагранжа:
\begin{equation}
\mathscr{L} = \Phi + \sum_i{\xi_i \left[N_i - \sum_\alpha N_{i, \alpha} \right]}
\label{eq:lagr}
\end{equation} 


Условие экстремума функции Лагранжа \eqref{eq:lagr}, учитывая соотношения \eqref{eq:hid}:
\[
\left\{
\begin{aligned}
& \pd{\mathscr{L}}{N_{i, \alpha}} = \mu_{i, \alpha} + N_{i, \alpha} \cdot RT \frac{1}{N_{i, \alpha}} - \xi_i = 0, \qquad \alpha=L, G \\
& \pd{\mathscr{L}}{\xi_i} = N_i - \sum_{\alpha} N_{i,\alpha} = 0 
\end{aligned}
\right.,\quad\forall i
\]

Или:
\begin{equation}
\left\{
\begin{aligned}
& \mu_{i, G}^0 + RT \ln y_i + RT - \xi_i = 0\\
& \mu_{i, L}^0 + RT \ln x_i + RT - \xi_i = 0\\
& N_i - \sum_{\alpha} N_{i,\alpha} = 0
\end{aligned}
\label{eq:lagr2}
\right.,\quad\forall i
\end{equation}

Вычитая в \eqref{eq:lagr2} первое уравнение из второго, получаем, что в состоянии фазового равновесия выполняется условие:

\begin{equation}
\frac{y_i}{x_i} = \exp \left(\frac{\mu_{i, L}^0 - \mu_{i, G}^0}{RT}\right) = K_i(p,T),
\label{eq:lnK}
\end{equation}
где $K_i(p,T) \equiv \dfrac{y_i}{x_i}$ --- константа равновесия для компонента $i$.


\begin{note}
Обратите внимание на то, что выражение \eqref{eq:lnK} можно было получить напрямую из условия $\mu_1 = \mu_2$.
\end{note}


\section{Постановка задачи и допущения}

В этом разделе рассматривается вспомогательная задача о нахождении фазового равновесия при фиксированных давлении и температуре. Цель этого раздела --- переформулировать проблему в форме задачи условной минимизации.

Рассматривается многокомпонентная система, которая может расслаиваться на несколько фаз. Часть из компонентов не участвует в фазовых превращениях и формирует инертные фазы (скелет). Другие компоненты могут потенциально образовывать три подвижные фазы: жидкую (нефтяная) фаза, помечаемая индексом $L$, газовую фазу (индекс $G$), и водную фазу (индекс $W$).

Предполагается следующее распределение компонентов по фазам:
\begin{itemize}
\item Один компонент $\mathrm{H_2O}$ --- может присутствовать в водной и газовой фазах.
\item Легкие нефти --- могут присутствовать в жидкой и газовых фазах. Всего таких компонентов --- $N_{LO}$. (например, $\mathrm{C_5H_{12}}$).
\item Тяжелые нефти --- могут присутствовать только в жидкой фазе. Всего --- $N_{HO}$ компонентов.
\item Нерастворимые газы --- могут присутствовать только в газовой фазе. Всего --- $N_{G}$ компонентов.
%\item $N_{S}$ компонентов --- инертные компоненты-фазы --- существуют каждая в своей фазе.
\end{itemize}

Кроме того инертные компоненты, не участвующие в фазовом равновесии, образуют каждая свою фазу.

Газы, растворимые в нефти, могут моделироваться как легкие нефти. При этом процессы разгазирования могут быть учтены подбором константы равновесия.

Вводятся обозначения (все молярные доли рассматриваются по отношению к количеству молей $N$ всей смеси):
\begin{itemize}
\item $\lambda, \omega, \gamma$ --- молярные доли в смеси жидкой, водной и газовой фаз соответственно. При этом $\lambda + \omega + \gamma \ne 1$, т.к. учитываются также инертные фазы.
\item $c_0$ - молярная доля $\mathrm{H_2O}$ в смеси
\item В водной фазе находится только $\mathrm{H_2O}$, а значит $\omega$ является также молярной долей $\mathrm{H_2O}$ в водной фазе
\item $c_i, \: i = 1 \dividesymbol N_{LO}$ --- молярные доли легких нефтей в смеси
\item $c_i, \: i = N_{LO} + 1 \dividesymbol N_{LO} + N_{HO}$ --- молярные доли тяжелых нефтей в смеси
\item $c_i, \: i = N_{LO} + N_{HO} + 1 \dividesymbol N_{LO} + N_{HO} + N_G$ --- молярные доли нерастворимых газов
\item $c_i, \: i = N_{LO} + N_{HO} + N_G + 1 \dividesymbol N_{LO} + N_{HO} + N_G + N_S$ --- молярные доли инертных компонентов, которые также являются молярными долями соответствующих фаз
\item $\lambda_i, \: i = 1 \dividesymbol N_{LO}$ --- молярные доли легких нефтей в жидкой фазе

\end{itemize}

В таблице \ref{t:compphases} приведен пример распределения компонентов по фазам. Компоненты $\mathrm{C_5H_{12}}$ и $\mathrm{C_{10}H_{22}}$ считаются легкими нефтями, компонент $\mathrm{C_{15}H_{32}}$ --- тяжелой нефтью. В смеси присутствуют три нерастворимых газа ($\mathrm{N_2}$, $\mathrm{CO_2}$ и $\mathrm{O_2}$), а также два инертных компонента-фазы ($\mathrm{Sand}$ и $\mathrm{Gran}$)

\begin{table}[ht!]
\centering
\begin{tabular}{c|c|c|c|c|c|c}
\multicolumn{1}{c}{$\diagdown$}&
\multicolumn{1}{c}{L (нефть)}&
\multicolumn{1}{c}{G (газ)}&
\multicolumn{1}{c}{W (вода)}&
\multicolumn{1}{c}{S (песчаник)}&
Gr (гранит)&
$\sum$\\
\cline{2-6}
$\mathrm{H_2O}$& & $c_0 - \omega$ & $\omega$ & & & $c_0$\\
\cline{1-6}
$\mathrm{C_5H_{12}}$& $\lambda_1$ & $c_1 - \lambda_1$ & & & & $c_1$\\
\cline{2-6}
$\mathrm{C_{10}H_{22}}$& $\lambda_2$ & $c_2 - \lambda_2$ & & & & $c_2$\\
\cline{1-6}
$\mathrm{C_{15}H_{32}}$& $c_3$ & & & & & $c_3$\\
\cline{1-6}
$\mathrm{N_2}$& & $c_4$ & & & & $c_4$\\
\cline{2-6}
$\mathrm{CO_2}$& & $c_5$ & & & & $c_5$\\
\cline{2-6}
$\mathrm{O_2}$& & $c_6$ & & & & $c_6$\\
\cline{1-6}
$\mathrm{Sand}$& & & & $c_7$ & & $c_7$\\
\cline{2-6}
$\mathrm{Gran}$& & & & & $c_8$ & $c_8$\\
\cline{1-6}
\multicolumn{1}{c}{$\sum$}&
\multicolumn{1}{c}{$\lambda$}&
\multicolumn{1}{c}{$\gamma$}&
\multicolumn{1}{c}{$\omega$}&
\multicolumn{1}{c}{$c_7$}&
\multicolumn{1}{c}{$c_8$}&
\multicolumn{1}{c}{$1$}\\
\end{tabular}
\caption{Пример распределения компонентов по фазам}
\label{t:compphases}
\end{table}

Требуется найти количество фаз, на которое расслоится смесь, молярные доли этих фаз и их состав, при известных молярных долях компонентов $c_i, \; i = 0 \dividesymbol N_{LO} + N_{HO} + N_{G} + N_{S}$ и констант фазового равновесия $K_i (p, T), i = 0 \dividesymbol N_{LO} $. Здесь неизвестными величинами являются молярная доля водной фазы $ \omega $ и молярные доли легких нефтей в жидкой фазе $ \lambda_i $ --- всего $N_{LO} + 1$ неизвестная.

В термоодинамическом плане указанную проблему можно сформулировать в виде задачи условной оптимизации потенциала Гиббса. Так как потенциал Гиббса указанной системы неизвестен, предлагается рассматривать подвижные фазы как идеальные растворы, с параметрами подобранными таким образом, чтобы получить заданные константы фазового равновесия. В этом случае состояние такой виртуальной системы, в котором достигается минимум потенциала Гиббса, будет совпадать с равновесным состоянием рассматриваемой смеси.

После исключения несущественных слагаемых (зависящих только от $p$ и $T$), молярный потенциал Гиббса виртуальной системы принимает вид:

\begin{equation}
\begin{aligned}
{\phi} &= \omega \varkappa_0(p,T) + \sum_{i=1}^{N_{LO}} \lambda_i \varkappa_i(p,T) + \sum_{i=1}^{N_{LO}} L(\lambda_i) - L(\lambda) + \\
&+ L(c_0 - \omega) + \sum_{i=1}^{N_{LO}} L(c_i - \lambda_i) - L(s - \lambda - \omega)
\end{aligned}
\label{eq:pproblem}
\end{equation}

Минимум функции $\phi$ ищется при ограничениях:

\begin{equation}
\left\{
\begin{aligned}
0& \leq \omega \leq c_0\\
0& \leq \lambda_i \leq c_i, \qquad i = 1,2,\dots,N_{LO}
\end{aligned}
\right.
\label{eq:restr}
\end{equation}

Здесь
\begin{itemize}
\item $\varkappa_i(p, T) = \ln K_i(p, T)$ --- логарифмы констант равновесия
\item $L(x) \equiv x \ln x$
\item $s \equiv \gamma + \lambda + \omega$
\end{itemize}

В общем случае $s \ne 1$, так как задача сформулирована для всех компонентов, а не отдельно для подвижных фаз.



\section{Энтальпия}
Фазовые переходы жидкость-газ в чистых веществах при фиксированном давлении происходят при температуре фазового перехода, определяемой этим давлением. Это обстоятельство приводит к двум проблемам:
\begin{itemize}
\item В некоторых случаях недостаточно знания молярных долей компонентов, давления и температуры для того, чтобы однозначно определить фазовый состав.
\item Зависимость фазовых составов от температуры может содержать разрывы первого рода.
\end{itemize}

Таким образом, для однозначного определения фазового состава требуется дополнительный параметр или соотношение. В данной работе в качестве такого параметра используется молярная энтальпия смеси, фигурирующая в уравнении теплового баланса.

В рассматриваемой задаче молярная энтальпия принимается в виде:
\begin{equation}
\eta = 
\sum_{i=0}^{N_{LO}+N_{HO}+N_{G}+N_{S}} c_i h_i(p,T) -\omega h_{0,G-W}(p,T) -\sum_{i=1}^{N_{LO}} \lambda_i h_{i,G-L}(p,T) 
,
\label{eq:defenth}
\end{equation}
где $h_{i,G-\alpha}(p,T)$ --- удельная молярная энтальпия парообразования (положительная величина), а $h_i(p,T)$ --- молярные энтальпии компонентов в ``основной'' фазе. Для $\mathrm{H_2O}$ и легких нефтей ``основной'' фазой является $G$.

С другой стороны, энтальпия может являться не вспомогательным, а определяющим параметром. Условия фазового равновесия системы (минимизация функционала \eqref{eq:pproblem} по неизвестным $\{\omega, \lambda_i\}$) и соотношение \eqref{eq:defenth} совместно рассматривается как система нелинейных уравнений для определения температуры и соответствующего этой температуре фазового равновесия. Эту систему можно трактовать следующим образом: требуется подобрать такую температуру смеси, чтобы после установления фазового равновесия ее молярная энтальпия равнялась заданной величине.

Отметим, что условие \eqref{eq:defenth} \emph{не является} ограничением в задаче \eqref{eq:pproblem}. Соотношение \eqref{eq:defenth} --- это уравнение на $T$, а блок фазового равновесия --- это неявная функция фазовых составов от температуры.


\section{Метод решения}

Как отмечалось во введении, целью настоящей работы являлось построение алгоритма, рассчитывыющего не только фазовое равновесие, но и производные по независимым переменным --- давлению, температуре, составу. Проблема состоит в том, что на поверхностях, где изменяется количество фаз, производные не определены. Более того, производные неопределены и в случае отсутствия в смеси того или иного компонента. Для преодоления этих трудностей предлагается произвести коррекцию функционала, состоящую во введении барьеров на границе области минимизации. Этот подход является аналогом метода логарифмических барьеров, который используется для решения задач условной оптимизации.

Барьеры невозможно ввести, если ограничение вырождается в равенство. Поэтому предлагается в начале производить коррекцию молярных долей активных компонентов $c_i \rightarrow c_i + \epsilon_i$. Это соответствует тому, что все компоненты всегда присутствуют при расчете фазового равновесия. После расчета фазового равновесия состав смеси корректируется таким образом, чтобы удалить неприсутствующие исходно компоненты и учесть все ограничения.

В соответствии с этими рассуждениями произведем коррекцию функционала $\phi$ \eqref{eq:pproblem} следующим образом:

\begin{equation}
\psi = \phi - \varepsilon \left[
\ln \omega + \ln (c_0 - \omega) + 
\sum_{i=1}^{N_{LO}} \Big( \ln \lambda_i
+ \ln (c_i - \lambda_i) \Big) 
\right]
\end{equation}

После такой коррекции строго внутри области будет существовать локальный минимум функционала $\psi$, находящийся вблизи искомого минимума функционала $\phi$.
Исходную задачу можно переформулировать в виде: найти минимум функции $\psi$ по переменным $\{\omega, \lambda_i\}$, при условии $\eta(\omega, \lambda_i, T) = h$, где $h$ --- заданное значение молярной энтальпии. Отметим, что последнее условие является не ограничением в задаче минимизации функционала $\psi$ (модельный потенциал Гиббса $\psi$ не минимизируется по температуре $T$), а дополнительным уравнением для определения температуры смеси. Таким образом получаем систему нелинейных уравнений:

\begin{equation}
0 = \mathbf{F}(\mathbf{x}) \equiv
\begin{cases}
 \pd{\psi(\mathbf{x})}{\omega}\\
 \pd{\psi(\mathbf{x})}{\lambda_i}\\
 \eta(\mathbf{x}) - h
\end{cases},
\label{eq:system}
\end{equation}
где $\mathbf{x} = (\omega, \lambda_i, T)$.

Система \eqref{eq:system} далее решается модифицированным методом Ньютона:

\begin{equation}
\left\{
\begin{aligned}
\Delta \mathbf{x}_k &= -\left(\pd{\mathbf{F(\mathbf{x}_k)}}{\mathbf{x}}\right)^{-1}
\mathbf{F}(\mathbf{x}_k)\\
\mathbf{x}_{k+1} &= \mathbf{x}_k + \alpha_k \Delta \mathbf{x}_k\\
\alpha_k &= \min(1, \beta \alpha_{k, \max})
\end{aligned}
\right.
\label{eq:newton}
\end{equation}

В методе \eqref{eq:newton} используется ограничитель шага $\alpha_k \leq 1$ для того, чтобы не выйти за границу области определения функции. Когда $\alpha_k = 1$ шаг происходит согласно обычному методу Ньютона. $\alpha_{k, \max} > 0$ --- это максимальное число, такое что $\mathbf{x}_k + \alpha_{k,\max} \Delta \mathbf{x}_{k}$ еще принадлежит области $\omega \in [0,c_0], \lambda_i \in [0, c_i]$, а также выполняется условие $T \in [T_{\min}, T_{\max}]$. Множитель $\beta < 1$ (типично $\beta = 0.9 \dividesymbol 0.99$) используется, чтобы новое приближение было строго внутри области.

Решение задачи можно аналитически продифференцировать по параметрам:

\begin{equation}
\begin{aligned}
&\mathbf{F}(\mathbf{x}(\boldsymbol \alpha), \boldsymbol \alpha) = 0\\
&\pd{\mathbf{F}(\mathbf{x}, \boldsymbol \alpha)}{\mathbf{x}} \pd{\mathbf{x}}{\boldsymbol \alpha} +
\pd{\mathbf{F}(\mathbf{x}, \boldsymbol \alpha)}{\boldsymbol \alpha} = 0\\
&\pd{\mathbf{x}}{\boldsymbol \alpha} =
-\left(\pd{\mathbf{F}(\mathbf{x}, \boldsymbol \alpha)}{\mathbf{x}}\right)^{-1}
\pd{\mathbf{F}(\mathbf{x}, \boldsymbol \alpha)}{\boldsymbol \alpha} \\
\end{aligned}
\end{equation}

\section{Результаты}

\newpage

\begin{thebibliography}{9}
    \addcontentsline{toc}{section}{\refname}
    \bibitem{Prigozhin} Пригожин И., Дефей Р. Химическая термодинамика //Новосибирск: СО изд.«Наука. – 1966.
\end{thebibliography}

\end{document}